\documentclass[a4paper]{article}

\def\nterm {Spring}
\def\nyear {2020}
\def\nlecturer {Dan Perelstein}
\def\ncourse {THP 350: Sound Design}

\RequirePackage{etex}
\makeatletter
\ifx \nauthor\undefined
  \def\nauthor{Daniel Moore}
\else
\fi

\author{Based on lectures by \nlecturer \\\small Notes taken by \nauthor}
\date{\nterm\ \nyear}

\usepackage{amsfonts}
\usepackage{amsmath}
\usepackage{amssymb}
\usepackage{amsthm}
\usepackage{booktabs}
\usepackage[makeroom]{cancel}
\usepackage{caption}
\usepackage{enumitem}
\usepackage{fancyhdr}
\usepackage{graphicx}
\usepackage{mathdots}
\usepackage{mathtools}
\usepackage{microtype}
\usepackage{multirow}
\usepackage{pdflscape}
\usepackage{siunitx}
\usepackage{textcomp}
\usepackage{tabularx}
\usepackage{titlesec}
\usepackage[normalem]{ulem}
\usepackage[all]{xy}
\usepackage{imakeidx}

\makeindex[intoc, title=Index]
\indexsetup{othercode={\lhead{\emph{Index}}}}
%\setcounter{secnumdepth}{4}

\titleformat{\paragraph}
{\normalfont\normalsize\bfseries}{\theparagraph}{1em}{}
\titlespacing*{\paragraph}
{0pt}{3.25ex plus 1ex minus .2ex}{1.5ex plus .2ex}

\setlength{\parindent}{0pt}
\setlength{\parskip}{1em}

\ifx \nextra \undefined
  \usepackage[pdftex,
    hidelinks,
    pdfauthor={Daniel Moore},
    pdfsubject={\ncourse},
    pdftitle={\ncourse},
  pdfkeywords={\nterm\ \nyear\ \ncourse}]{hyperref}
  \title{\ncourse}
\else
  \usepackage[pdftex,
    hidelinks,
    pdfauthor={Daniel Moore},
    pdfsubject={\ncourse\ (\nextra)},
    pdftitle={\ncourse\ (\nextra)},
  pdfkeywords={\nterm\ \nyear\ \ncourse\ \nextra}]{hyperref}

  \title{\ncourse \\ {\Large \nextra}}
  \renewcommand\printindex{}
\fi

\pagestyle{fancyplain}
\ifx \ncoursehead \undefined
\def\ncoursehead{\ncourse}
\fi

\lhead{\emph{\nouppercase{\leftmark}}}
\ifx \nextra \undefined
  \rhead{
    \ifnum\thepage=1
    \else
      \ncoursehead
    \fi}
\else
  \rhead{
    \ifnum\thepage=1
    \else
      \ncoursehead \ (\nextra)
    \fi}
\fi

\newcommand*{\Cdot}{{\raisebox{-0.25ex}{\scalebox{1.5}{$\cdot$}}}}
\newcommand{\mbeq}{\overset{!}{=}}
\newcommand{\vhat}[1]{\vec{\hat{#1}}}
\newcommand{\x}{\vhat{x}}
\newcommand{\y}{\vhat{y}}
\newcommand{\z}{\vhat{z}}
\newcommand{\del}{\vec{\nabla}}
\ifx \nhtml \undefined
\else
  \renewcommand\printindex{}
  \DisableLigatures[f]{family = *}
  \let\Contentsline\contentsline
  \renewcommand\contentsline[3]{\Contentsline{#1}{#2}{}}
  \renewcommand{\@dotsep}{10000}
  \newlength\currentparindent
  \setlength\currentparindent\parindent

  \newcommand\@minipagerestore{\setlength{\parindent}{\currentparindent}}
  \usepackage[active,tightpage,pdftex]{preview}
  \renewcommand{\PreviewBorder}{0.1cm}

  \newenvironment{stretchpage}%
  {\begin{preview}\begin{minipage}{\hsize}}%
    {\end{minipage}\end{preview}}
  \AtBeginDocument{\begin{stretchpage}}
  \AtEndDocument{\end{stretchpage}}

  \newcommand{\@@newpage}{\end{stretchpage}\begin{stretchpage}}

  \let\@real@section\section
  \renewcommand{\section}{\@@newpage\@real@section}
  \let\@real@subsection\subsection
  \renewcommand{\subsection}{\@ifstar{\@real@subsection*}{\@@newpage\@real@subsection}}
\fi
\ifx \ntrim \undefined
\else
  \usepackage{geometry}
  \geometry{
    papersize={379pt, 699pt},
    textwidth=345pt,
    textheight=596pt,
    left=17pt,
    top=54pt,
    right=17pt
  }
\fi

\ifx \nisofficial \undefined
\let\@real@maketitle\maketitle
\renewcommand{\maketitle}{\@real@maketitle\begin{center}\begin{minipage}[c]{0.9\textwidth}\centering\footnotesize These notes are not endorsed by the lecturers, and I have modified them (often significantly) after lectures. They are nowhere near accurate representations of what was actually lectured, and in particular, all errors are almost surely mine.\end{minipage}\end{center}}
\else
\fi

% Theorems
\theoremstyle{definition}
\newtheorem*{aim}{Aim}
\newtheorem*{ans}{Answer}
\newtheorem*{axiom}{Axiom}
\newtheorem*{claim}{Claim}
\newtheorem*{cor}{Corollary}
\newtheorem*{conjecture}{Conjecture}
\newtheorem*{defi}{Definition}
\newtheorem*{eg}{Example}
\newtheorem*{ex}{Exercise}
\newtheorem*{fact}{Fact}
\newtheorem*{law}{Law}
\newtheorem*{notation}{Notation}
\newtheorem*{prop}{Proposition}
\newtheorem*{post}{Postulate}
\newtheorem*{question}{Question}
\newtheorem*{problem}{Problem}
\newtheorem*{rrule}{Rule}
\newtheorem*{thm}{Theorem}
\newtheorem*{assumption}{Assumption}

\newtheorem*{remark}{Remark}
\newtheorem*{warning}{Warning}
\newtheorem*{exercise}{Exercise}

\newtheorem{nthm}{Theorem}[section]
\newtheorem{nlemma}[nthm]{Lemma}
\newtheorem{nprop}[nthm]{Proposition}
\newtheorem{ncor}[nthm]{Corollary}


\renewcommand{\labelitemi}{--}
\renewcommand{\labelitemii}{$\circ$}
\renewcommand{\labelenumi}{(\roman{*})}

\let\stdsection\section
\renewcommand\section{\newpage\stdsection}

% Strike through
\def\st{\bgroup \ULdepth=-.55ex \ULset}


%%%%%%%%%%%%%%%%%%%%%%%%%
%%%%% Maths Symbols %%%%%
%%%%%%%%%%%%%%%%%%%%%%%%%

% Brackets
\newcommand{\abs}[1]{\left\lvert #1\right\rvert}
\newcommand{\expval}[3]{\left\langle #1 \left\vert #2 \right\vert #3 \right\rangle}
\newcommand{\norm}[1]{\left\lVert #1\right\rVert}
\newcommand{\normalorder}[1]{\mathop{:}\nolimits\!#1\!\mathop{:}\nolimits}
\newcommand{\tv}[1]{|#1|}
\renewcommand{\vec}[1]{\boldsymbol{\mathbf{#1}}}
\newcommand{\matr}[1]{\mathbf{#1}}
\newcommand{\exval}[1]{\left\langle #1\right\rangle}

% not-math
\newcommand{\bolds}[1]{{\bfseries #1}}
\newcommand{\cat}[1]{\mathsf{#1}}
\newcommand{\ph}{\,\cdot\,}
\newcommand{\term}[1]{\emph{#1}\index{#1}}
\newcommand{\phantomeq}{\hphantom{{}={}}}

\let\Im\relax
\let\Re\relax

\newcommand\qedshift{\vspace{-17pt}}
\newcommand\fakeqed{\pushQED{\qed}\qedhere}

\def\Xint#1{\mathchoice
   {\XXint\displaystyle\textstyle{#1}}%
   {\XXint\textstyle\scriptstyle{#1}}%
   {\XXint\scriptstyle\scriptscriptstyle{#1}}%
   {\XXint\scriptscriptstyle\scriptscriptstyle{#1}}%
   \!\int}
\def\XXint#1#2#3{{\setbox0=\hbox{$#1{#2#3}{\int}$}
     \vcenter{\hbox{$#2#3$}}\kern-.5\wd0}}
\def\ddashint{\Xint=}
\def\dashint{\Xint-}

\newcommand\separator{{\centering\rule{2cm}{0.2pt}\vspace{2pt}\par}}

\newenvironment{own}{\color{gray!70!black}}{}

\newcommand\makecenter[1]{\raisebox{-0.5\height}{#1}}

\mathchardef\mdash="2D

\newenvironment{significant}{\begin{center}\begin{minipage}{0.9\textwidth}\centering\em}{\end{minipage}\end{center}}
\DeclareRobustCommand{\rvdots}{%
  \vbox{
    \baselineskip4\p@\lineskiplimit\z@
    \kern-\p@
    \hbox{.}\hbox{.}\hbox{.}
  }}
\DeclareRobustCommand\tph[3]{{\texorpdfstring{#1}{#2}}}
\makeatother


\begin{document}
\maketitle

\tableofcontents

\section{Sound System Overview \& Signal Flow}

\subsection{What is a sound system?}
The sound systm is the selection and configuration of the loudspeakers,
microphones, equipment, and specification of connections between all the
elements. The sound system is designed uniquely for every production, based on
and influenced by:
\begin{itemize}
	\item Text of the play
	\item Director/choreographer's vision
	\item Venue
		\begin{itemize}
			\item Layout
			\item Acoustics
		\end{itemize}
	\item Scenic design
		\begin{itemize}
			\item What sounds make sense given the scene being
				seen?
		\end{itemize}
	\item Conventions and audience expectations
		\begin{itemize}
			\item Doing something different or unique is allowed,
				but it should be a thought-through choice
		\end{itemize}
	\item Available Inventory
	\item Constraints from other departments
\end{itemize}
Every sound system will have the same basic structure:
\begin{enumerate}
	\item Sound source(s)
		\begin{itemize}
			\item Mics
			\item Keyboards
			\item Synths
			\item Guitars
			\item DI Boxes
				\begin{itemize}
					\item Takes an error-prone signal and
						makes it stronger, clearer, and
						protected from interference
					\item Converts from an unbalanced
						signal to a balanced signal
				\end{itemize}
			\item Computers
			\item CD Players
			\item Etc.
		\end{itemize}
	\item Mixer (``console,'' ``desk,'' ``board'')
		\begin{itemize}
			\item Pre-amp (``head amp'') is the first step of this stage
		\end{itemize}
	\item Power amplifiers (``amps'')
	\item Loudspeakers
	\item Audience/performer's ears
\end{enumerate}
There are some common variations on this system:
\begin{itemize}
	\item Combo amps: Items ii-iv are combined in one place
	\item Active/powered loudspeaker: A loudspeaker with an amplifier
		built-in
		\begin{itemize}
			\item Might have a power cable, indicator light,
				screen, etc.
		\end{itemize}
\end{itemize}

\subsection{Signal Levels and Amplification}
Another ``narrative'' of a sound system is that of a quiet sound becoming a
much larger/louder sound. Each step in the system is a way to step up the
signal level from the quietest to the loudest.

There are basically three signal levels:
\begin{enumerate}
	\item Mic/Instrument Level
		\begin{itemize}
			\item Level of the signal coming directly from a
				mic/input
			\item Approx. 2-20 mV
		\end{itemize}
	\item Line Level
		\begin{itemize}
			\item Level of the signal coming from the console after
				pre-amplification (or the gain stage)
			\item Approx. 2-20 V
		\end{itemize}
	\item Speaker Level
		\begin{itemize}
			\item Level of the signal coming to the speakers after
				power amplification
			\item Approx. 10-50 V
		\end{itemize}
\end{enumerate}



\section{Engineering Overview}
\subsection{A Basic Layout Diagram}
\includegraphics[width=\textwidth]{BasicDiagram.png}

\subsection{Review}
\subsubsection{Sound Systems Design Fundamentals}
There are two goals in sound system design:
\begin{enumerate}
	\item Reasonable choices in terms of imaging
		\begin{itemize}
			\item Imaging --- the idea that we can create a third,
				imaginary sound using two different speakers.
				The illusion of stereo
		\end{itemize}
	\item Even coverage for every audience member
		\begin{itemize}
			\item This can often be improved using fill speakers
				for the first few rows
		\end{itemize}
\end{enumerate}

\subsubsection{Dispersion}
Sound varies based on your position relative to a loudspeaker in 2 ways:
\begin{enumerate}
	\item Distance from a loudspeaker
		\begin{itemize}
			\item The farther you are from a speaker, the quieter
				the sound is
		\end{itemize}
	\item Axial position relative to speaker (ie. angle from the center
		axis)
		\begin{itemize}
			\item The farther off-axis you go, the quieter the
				sound is, and the more you lose higher
				frequencies
		\end{itemize}
\end{enumerate}

\subsection{Types of Connectors}
\begin{enumerate}
	\item XLR Microphone Cable\\
		\includegraphics[width=0.5\textwidth]{XLR.jpg}\\
		In audio, we use 3-pin XLRs (one pin is positive, one is
		negative, one is ground). One cool thing about XLR connectors
		is that they usually lock.
	\item 1/4'' Instrument Cable\\
		\includegraphics[width=0.5\textwidth]{Quarter.jpeg}\\
		1/4'' cables can either be balanced or unbalanced. Balanced
		cables have three conductors (tip, ring, sleeve, or TRS, for +,
		-, and ground, respectively), and
		unbalanced cables have two conductors (tip and sleeve, or TS,
		for + and -, respectively).
	\item 1/4'' Speaker Cable\\
		These look basically the same, but they're thicker (different
		gauge, can be used to pass more voltage).
	\item SpeakOn Cable (NL2, NL4, NL8)\\
		\includegraphics[width=0.5\textwidth]{NL2.jpeg}\\
		Thick balanced cables that can lock. They carry more voltage
		than 1/4'' cables.
	\item Bare Wire Speaker Cable\\
		Balanced or unbalanced, no connector, it's just the wires
		straight. Usually used in situations where there's no movement
		or chance of tripping.
	\item RCA Cable\\
		\includegraphics[width=0.5\textwidth]{RCA.jpeg}\\
		Low-voltage unbalanced cable.
	\item 1/8'' to dual XLR adapter (aka ``iPod Cable'')\\
		\includegraphics[width=0.5\textwidth]{18XLR.jpeg}\\
\end{enumerate}

\subsection{Audio Channels and Common Configurations}
Common audio channel configurations:
\begin{itemize}
	\item Mono (one output to one speaker)
	\item Dual mono (same signal to both speakers, or totally different
		signals to each, like two different mono signals)
	\item Stereo (signal shared across two speakers---panning is possible)
	\item 5.1, 7.1, etc. (usually just for film---we sometimes replicate
		similar things in theatre, but industry film standards don't
		really work for us)
\end{itemize}

\section{Design of Sound Systems II: Imagining}
\subsection{Imaging in Horizontal and Vertical Planes}
\subsection{Delay Times and Haas Effect}

\section{Design of Sound Systems III: Sub-Systems}
\subsection{Phase, Phasing, Polarity}
\subsection{Surrounds, Subwoofers, Sub-Systems, and Unconventional Stagings}

\end{document}
